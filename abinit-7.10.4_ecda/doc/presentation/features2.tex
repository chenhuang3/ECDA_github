\newpage
\section*{
\vskip 1em
ABINIT : 6. MANY CAPABILITIES (II)}

\vskip 0.5em
\parbox{0.85\linewidth}{
\Large{

\begin{itemize}
 \item
 {\black{~}}{\green responses} to atomic displacements (even at non-zero wavevectors,
 without need of supercells !)
 and to homogeneous electric fields,
 within Density-Functional Perturbation Theory:

 dielectric tensor, Born effective charges,
 dynamical matrices at any wavevector, phonon frequencies, force constants
 phonon density of states, thermodynamic properties in the quasi-harmonic approximation
 \item
 {\black{~}}{\green responses to strain perturbations}: elastic constants, piezoelectric coefficients.
 \item
 {\black{~}}{\green non-linear responses} thanks to the 2n+1 theorem of perturbation theory :
 at present, electro-optic coefficients, Raman cross-sections.
 \item
 {\black{~}} susceptibility matrix by sum over states
 \item
 {\black{~}}{\green excited states} of atoms and molecules within Time Dependent-DFT or Delta SCF
 \item
 {\black{~}} frequency-dependent conductivity in the RPA (Kubo-Greenwood)
 \item
 {\black{~}}{\green exact exchange and RPA+} calculation of total energies
 (one k-point, post-LDA or post-GGA, not yet available for spin-polarized systems or spinor wavefunctions)
 \item
 {\black{~}}{\green GW calculation of excited states}, also with PAW,
not yet available for spinor wavefunctions
 \item
 {\black{~}}{\green MPI parallelisation} of ground-state and response-function calculations over k-points,
 spins and bands, {\green MPI parallelisation} of FFT grid and planewave operations
\end{itemize}
}}
