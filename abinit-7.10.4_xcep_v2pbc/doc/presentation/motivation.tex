\section*{
\vskip 1em
FIRST-PRINCIPLES STUDIES OF MATERIALS : HOW \\
          TO ORGANIZE THE SOFTWARE DEVELOPMENT ? }

\vskip 1em
\parbox{0.92\linewidth}{
\Large{
 The state-of-the-art is in constant progression ... \\}
\large{
 (the following list is not exhaustive, and is biased towards pseudopotentials)
}
\Large{
\begin{itemize}
 \item
  1985 Car-Parinello technique
 \item
  1985 GW calculations
 \item
  1987 Linear-response approach to phonons, dielectric tensor ...
 \item
  1991 Ultrasoft pseudopotentials
 \item
  1992 Berry phase approach to polarization
 \item
  1994 Projector Augmented Waves
 \item
  1996 Time-dependent DFT for excited states
 \item
  1997 Maximally-localized Wannier functions
 \item
  1998 Bethe-Salpether equation
 \item
  2002 Treatment of finite electric fields
\end{itemize}
 ... not even mentioning other advances for which one cannot clearly isolate
 a date (parallelism, order(N) scaling, DMFT, ...).\\
 Obviously, the frontier of research has moved a lot. \\
 This trend will continue !  \\
 In order to stay up-to-date, softwares will have to include
 more and more ``basics''.\\
}

\Huge{
\begin{itemize}
 \item[\Large\green\ding{52}] Software engineering concepts are needed.
 \item[\Large\green\ding{52}] Collaborative effort of groups is needed.
\end{itemize}
}}
