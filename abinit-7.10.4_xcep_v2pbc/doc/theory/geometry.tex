%\title{ABINIT data structures and their theoretical justifications}

\documentclass{article}
%\documentstyle{article}

\title{Geometric considerations \\
       {\normalsize{(X. Gonze, Y. Suzukawa, M. Mikami)}}}

\begin{document}
\maketitle

\section{Real space}
\hspace*{\parindent}
\vskip 1em
* The three primitive translation vectors are {\bf R}$_{1p}$,
{\bf R}$_{2p}$, {\bf R}$_{3p}$.
% CAUTION : $\vec{R}_{1p}$ etc. does not work due to the specification.
% See the TtH manual.

Representation in Cartesian coordinates (atomic units):
%\[\vec{R}_{1p} \rightarrow rprimd(1:3,1)\]
\[{\bf R}_{1p} \rightarrow {\tt rprimd(1:3,1)}\]
\[{\bf R}_{2p} \rightarrow {\tt rprimd(1:3,2)}\]
\[{\bf R}_{3p} \rightarrow {\tt rprimd(1:3,3)}\]

Related input variables : {\tt acell,rprim,angdeg}

\vskip 1em
* Atomic positions are specified by the coordinates {\bf x}$_{\tau}$
for $\tau=1 \dots N_{atom}$ where $N_{atom}$ is the member of atoms.

Representation in reduced coordinates
\begin{eqnarray*}
{\bf x}_{\tau} &=& x^{red}_{1\tau} \cdot {\bf R}_{1p}
 + x^{red}_{2\tau} \cdot {\bf R}_{2p}
 + x^{red}_{3\tau} \cdot {\bf R}_{3p} \\
 \tau &\rightarrow& {\tt iatom} \\
 N_{atom} &\rightarrow& {\tt natom} \\
 x^{red}_{1\tau} &\rightarrow& {\tt xred(1,iatom)} \\
 x^{red}_{2\tau} &\rightarrow& {\tt xred(2,iatom)} \\
 x^{red}_{3\tau} &\rightarrow& {\tt xred(3,iatom)}
\end{eqnarray*}

Related input variables : {\tt xangst,xcart,xred}

\vskip 1em
* The volume of the primitive unit cell is
\begin{eqnarray*}
\Omega_{O{\bf r}} &=& {\bf R}_1 \cdot ({\bf R}_2 \times {\bf R}_3)  \\
\Omega_{O{\bf r}} &\rightarrow& {\tt ucvol} \,{\rm (unit \ cell \ volume)}
\end{eqnarray*}
Computed in {\tt metric.f}

\vskip 1em
* The scalar products in the reduced representation are valuated thanks to
\[ {\bf r} \cdot {\bf r'} =\left(
\begin{array}{ccc}
r^{red}_{1} & r^{red}_{2} & r^{red}_{1}
\end{array}
\right)
\left(
\begin{array}{ccc}
{\bf R}_{1p} \cdot {\bf R}_{1p} & {\bf R}_{1p} \cdot {\bf R}_{2p} &
{\bf R}_{1p} \cdot {\bf R}_{3p} \\
{\bf R}_{2p} \cdot {\bf R}_{1p} & {\bf R}_{2p} \cdot {\bf R}_{2p} &
{\bf R}_{2p} \cdot {\bf R}_{3p} \\
{\bf R}_{3p} \cdot {\bf R}_{1p} & {\bf R}_{3p} \cdot {\bf R}_{2p} &
{\bf R}_{3p} \cdot {\bf R}_{3p}
\end{array}
\right)
\left(
\begin{array}{c}
r^{red \prime}_{1} \\
r^{red \prime}_{2} \\
r^{red \prime}_{3}
\end{array}
\right) \]

that is
${\bf r} \cdot {\bf r'} = \sum_{ij} r^{red}_{i} {\bf R}^{met}_{ij}
r^{red \prime}_{j}$

where ${\bf R}^{met}_{ij}$ is the metric tensor in real space :

\[{\bf R}^{met}_{ij} \rightarrow {\tt rmet(i,j)}\]

Computed in {\tt metric.f}.

\vskip 1em
\section{Reciprocal space}
\hspace*{\parindent}
\vskip 1em
* The three primitive translation vectors in reciprocal space are
${\bf G}_{1p}$,${\bf G}_{2p}$,${\bf G}_{3p}$ (computed in {\tt metric.f})

\begin{eqnarray*}
{\bf G}_{1p}&=&\frac{1}{\Omega_{O{\bf r}}}({\bf R}_{2p}\times{\bf R}_{3p})
 \rightarrow {\tt gprimd(1:3,1)} \\
{\bf G}_{2p}&=&\frac{1}{\Omega_{O{\bf r}}}({\bf R}_{3p}\times{\bf R}_{1p})
 \rightarrow {\tt gprimd(1:3,2)} \\
{\bf G}_{3p}&=&\frac{1}{\Omega_{O{\bf r}}}({\bf R}_{1p}\times{\bf R}_{2p})
 \rightarrow {\tt gprimd(1:3,3)}
\end{eqnarray*}

This definition is such that \({\bf G}_{ip}\cdot{\bf R}_{jp}=\delta_{ij}\)

[WARNING: often, a factor of $2\pi$ is present in definition of
 ${\bf G}_{ip}$, but not here, for historical reasons.]

\vskip 1em
* Reduced representation of vectors (K) in reciprocal space

\({\bf K}=K^{red}_{1}{\bf G}_{1p}+K^{red}_{2}{\bf G}_{2p}
+K^{red}_{3}{\bf G}^{red}_{3p} \rightarrow
(K^{red}_{1},K^{red}_{2},K^{red}_{3}) \)

e.g. the reduced representation of ${\bf G}_{1p}$ is (1,0,0).

\vskip 1em
* The reduced representation of the vectors of the reciprocal space
lattice is made of triplets of integers.

\vskip 1em
*The scalar products in the reduced representation are evaluated
thanks to

\[ {\bf K} \cdot {\bf K'}=\left(
\begin{array}{ccc}
K^{red}_{1} & K^{red}_{2} & K^{red}_{1}
\end{array}
\right)
\left(
\begin{array}{ccc}
{\bf G}_{1p} \cdot {\bf G}_{1p} & {\bf G}_{1p} \cdot {\bf G}_{2p}
& {\bf G}_{1p} \cdot {\bf G}_{3p} \\
{\bf G}_{2p} \cdot {\bf G}_{1p} & {\bf G}_{2p} \cdot {\bf G}_{2p}
& {\bf G}_{2p} \cdot {\bf G}_{3p} \\
{\bf G}_{3p} \cdot {\bf G}_{1p} & {\bf G}_{3p} \cdot {\bf G}_{2p}
& {\bf G}_{3p} \cdot {\bf G}_{3p}
\end{array}
\right)
\left(
\begin{array}{c}
K^{red \prime}_{1} \\
K^{red \prime}_{2} \\
K^{red \prime}_{3}
\end{array}
\right) \]

that is \({\bf K} \cdot {\bf K'} =
\sum_{ij} K^{red}_{i}{\bf G}^{met}_{ij}K^{red \prime}_{j}\)

where ${\bf G}^{met}_{ij}$ is the metric tensor in reciprocal space :
\[{\bf G}^{met}_{ij} \rightarrow {\tt gmet(i,j)}\]
(computed in {\tt metric.f}).

\section{Symmetries}
\hspace*{\parindent}
\vskip 1em
* A symmetry operation in real space sends the point ${\bf r}$ to the point
${\bf r'}={\bf S_t}\{{\bf r}\}$ whose coordinates are
$({\bf r'})_{\alpha}=\sum_{\beta}S_{\alpha \beta}r_{\beta}
+ t_{\alpha}$ (Cartesian coordinates).

\vskip 1em
* The symmetry operations that preserves the crystalline structure are
those that send every atom location on an atom location with the same
atomic type.

\vskip 1em
* The application of a symmetry operation to a function of spatial
coordinates ${\bf V}$ is such that :

\[({\bf S_t}{\bf V})({\bf r})={\bf V}(({\bf S_t})^{-1}\{{\bf r}\}\]
\[({\bf S_t})^{-1}\{{\bf r}\}=\sum_{\beta} S^{-1}_{\alpha \beta}
 (r_{\beta}-t_{\beta})\]

\vskip 1em
* For each symmetry operation,$isym=1 \dots nsym$, the $3\times3$
${\bf S}^{red}$ matrix is stored in {\tt symrel(:,:,isym)}.

[in reduced coordinates :
 $r'^{red}_{\alpha}=\sum_{\beta}S^{red}_{\alpha \beta}
 r^{red}_{\beta}+t^{red}_{\beta}$ ]

and the vector ${\bf t}^{red}$ is stored in {\tt tnons (:,isym)}.

\vskip 1em
* The conversion between reduced coordinates and Cartesian coordinates is
$r'_{\gamma}=\sum_{\alpha \beta}(R_{\alpha p})_{\gamma}
[S^{red}_{\alpha \beta} r^{red}_{\beta}+t^{red}_{\alpha}]$

with $[{\rm as} \ G_{ip} \cdot R_{jp}=\delta_{ij}]$
\[
r_{\delta}=\sum_{\alpha}(R_{\alpha p})_\delta
r^{red}_{\alpha} \rightarrow
\sum_{\beta}(G_{\beta p})_{\delta} r_{\delta}=r^{red}_{\beta}
\].

So \[S_{\gamma \delta}=\sum_{\alpha \beta}(R_{\alpha p})_{\gamma}
S^{red}_{\alpha \beta}(G_{\beta p})_{\gamma}\]

\end{document}






