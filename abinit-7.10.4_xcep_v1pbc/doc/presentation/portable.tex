\newpage
\section*{
 \vskip 1em
 ABINIT : 4. PORTABLE }

\vskip 1em
\parbox{0.92\linewidth}{
\Large{
\begin{itemize}
 \item
 Is ABINIT platform-dependent ?
 \item
 How difficult is it to install ABINIT on a platform ?
\end{itemize}

\vskip 1em
 The installation procedure is based on the autotools, and might be as simple as ``configure ; make''. \\
 In the worst case, one machine-dependent file must be set up once,  \\
  and will work for the next installations on the same machine. \\
  It contains the compiler name and options, library locations ...  \\

 Over the years, ABINIT has been installed on :
}
}

\large{
\vskip 0.5em
\begin{itemize}
 \item[\green\ding{52}]
 Intel - Pentium under Linux (IFC, PGI, g95, Pathscale, Fujitsu, NAG compilers)
 \item[\green\ding{52}]
 Intel - Pentium under Windows NT, Windows 98
 \item[\green\ding{52}]
 Opteron
 \item[\green\ding{52}]
 Itanium 2
 \item[\green\ding{52}]
 Compaq - alpha EV68, EV67, EV6, EV56, under UNIX + Linux
 \item[\green\ding{52}]
 SGI - Origin 200/2000, Altix/Itanium II, under UNIX
 \item[\green\ding{52}]
 IBM - Power IV, III+, Power II, under UNIX
 \item[\green\ding{52}]
 Sun - Ultrasparc III and II, under UNIX
 \item[\green\ding{52}]
 FUJITSU - VPP, under UNIX
 \item[\green\ding{52}]
 NEC - SX4, SX5, under UNIX
 \item[\green\ding{52}]
 MAC - Power G5, under MacOS X
\end{itemize}

}
